\documentclass[specification,annotation]{itmo-student-thesis}
% \documentclass[specification,annotation]{itmo-student-thesis}

%% Опции пакета:
%% - specification - если есть, генерируется задание, иначе не генерируется
%% - annotation - если есть, генерируется аннотация, иначе не генерируется
%% - times - делает все шрифтом Times New Roman, требует пакета pscyr.

%% Делает запятую в формулах более интеллектуальной, например:
%% $1,5x$ будет читаться как полтора икса, а не один запятая пять иксов.
%% Однако если написать $1, 5x$, то все будет как прежде.
\usepackage{icomma}

%% Данные пакеты необязательны к использованию в бакалаврских/магистерских
%% Они нужны для иллюстративных целей
%% Начало
\usepackage{tikz}
\usetikzlibrary{arrows}
\usepackage{filecontents}

% Links to papers that are going to be translated to proper format:
% (1) https://bitcoin.org/bitcoin.pdf
% (2) http://vukolic.com/iNetSec_2015.pdf
% (3) https://iohk.io/research/papers/#9BKRHCSI
% (4) http://cs.umd.edu/~amiller/permacoin.pdf
% (5) https://github.com/input-output-hk/plutus-prototype/blob/master/docs/spec/Formal%20Specification%20of%20the%20Plutus%20Language%20-%20McAdams.pdf

\begin{filecontents}{bachelor-thesis.bib}
@inproceedings{ example-english,
    year        = {2015},
    booktitle   = {Proceedings of IEEE Congress on Evolutionary Computation},
    author      = {Maxim Buzdalov and Anatoly Shalyto},
    title       = {Hard Test Generation for Augmenting Path Maximum Flow
                   Algorithms using Genetic Algorithms: Revisited},
    pages       = {2121-2128},
    langid      = {english}
}

@article{ example-russian,
    author      = {Максим Викторович Буздалов},
    title       = {Генерация тестов для олимпиадных задач по программированию
                   с использованием генетических алгоритмов},
    journal     = {Научно-технический вестник {СПбГУ} {ИТМО}},
    number      = {2(72)},
    year        = {2011},
    pages       = {72-77},
    langid      = {russian}
}

@article{ unrestricted-jump-evco,
    author      = {Maxim Buzdalov and Benjamin Doerr and Mikhail Kever},
    title       = {The Unrestricted Black-Box Complexity of Jump Functions},
    journal     = {Evolutionary Computation},
    year        = {2016},
    note        = {Accepted for publication},
    langid      = {english},
    keywords    = {index:wos, index:scopus, labauthor:buzdalov}
}
\end{filecontents}
%% Конец

%% Указываем файл с библиографией.
\addbibresource{bachelor-thesis.bib}

\begin{document}

\studygroup{M3438}
\title{Тайм-трекинг пользовательских действий на распределенной блокчейн-сети}
\author{Волхов Михаил Александрович}{Волхов М.А.}
\supervisor{Штукенберг Дмитрий Григорьевич}{Штукенберг Д.Г.}{специалист}{тьютор кафедры КТ Университета ИТМО}
\publishyear{2017}
%% Дата выдачи задания. Можно не указывать, тогда надо будет заполнить от руки.
% \startdate{01}{сентября}{2016}
%% Срок сдачи студентом работы. Можно не указывать, тогда надо будет заполнить от руки.
% \finishdate{31}{мая}{2017}
%% Дата защиты. Можно не указывать, тогда надо будет заполнить от руки.
% \defencedate{15}{июня}{2015}

% \addconsultant{}{канд. физ.-мат. наук, без звания}
\addconsultant{Йонн Мостовой}{без степени, без звания}

%% Задание и аннотация
%%% Техническое задание и исходные данные к работе
\technicalspec{Предложить дизайн системы для тайм-трекинга
  многопользовательских действий с помощью блокчейн-технологии.  }

%%% Содержание выпускной квалификационной работы (перечень подлежащих разработке вопросов)
\plannedcontents{
  \begin{enumerate}
  \item Постановка задачи и обзор предметной области.
  \item Анализ изменений базовой технологии (криптовалюты), связанный
    с разработкой.
  \item Разработка архитектуры.
  \end{enumerate}}

%Алгоритм должен поддерживать многопользовательские действия. Должна быть разрабтана формализация социальных контрактов и система оценки пользовательских действий на основании данных тайм-трекинга.

%%% Исходные материалы и пособия
\plannedsources{
  \begin{enumerate}
  \item Eli Ben-Sasson, Alessandro Chiesa, Christina Garman, Matthew Green, Ian Miers, Eran Tromer, Madars Virza, Zerocash: Decentralized Anonymous Payments from Bitcoin, proceedings of the IEEE Symposium on Security \& Privacy (Oakland) 2014, 459-474, IEEE, 2014;
  \item Aggelos Kiayias, Alexander Russell, Bernardo David, Roman Oliynykov, Ouroboros: A Provably Secure Proof-of-Stake Blockchain Protocol, 2017.
  \item Satoshi Nakamoto. Bitcoin: A peer-to-peer electronic cash system, http://bitcoin.org/bitcoin.pdf, 2008.
  \end{enumerate}
}

%%% Календарный план
\addstage{Ознакомление с имеющимися решениями в сфере тайм-трекинга}{02.2017}
\addstage{Ознакомление с практиками построения современных криптовалют}{02.2017}
\addstage{Реализация общего вида алгоритма}{03.2017}
\addstage{Написание пояснительной записки с проработанным формальным фреймворком}{04.2015}
\addstage{Доработка записки и доказательств}{05.2015}

%%% Цель исследования
\researchaim{Разработка тайм-трекинг системы с мультипользовательскими контрактами как надстройки над криптовалютой}

%%% Задачи, решаемые в ВКР
\researchtargets{\begin{enumerate}
  \item Соответствие алгоритма требованиям архитектуры блокчейн.
  \item Гибкий интерфейс: возможность реализации мобильного клиента для использования с алгоритмом.
  \item Формализация контрактной системы должна покрывать минимум повторяющиеся контракты с фиксированным временем в период.
  \item Алгоритм должен быть спроектирован так, чтобы не нарушать гарантий безопасности базовой системы.
\end{enumerate}}

%%% Использование современных пакетов компьютерных программ и технологий
\advancedtechnologyusage{Была использована система компьютерной
  верстки \LaTeX, а в рамках нее следующие пакеты, в порядке появления
  в стилевом файле: babel, csquotes, geometry, amsmath, amssymb,
  amsthm, amsfonts, amsextra, graphicx, xcolor, colortbl, tabu,
  caption, floatrow, algorithm, algorithmicx, algpseudocode, enumitem,
  setspace, biblatex (а именно biber), lastpage, totcount, longtable,
  listings, chngcntr, titlesec, titletoc, ifpdf. Набор текста
  производился в свободном редакторе текста emacs.}

%%% Краткая характеристика полученных результатов
\researchsummary{ В результате данной работы была разработана
  архитектура тайм-трекера, являющаяся надстройкой над протоколом
  криптовалюты Ouroboros. Решение имеет поддержку
  многопользовательских действий и контрактов. Также были
  проанализированы модификации Ouroboros, необходимые для построения
  решения. }

%%% Гранты, полученные при выполнении работы
\researchfunding{Грантов при выполнении работы получено не было.}

%%% Наличие публикаций и выступлений на конференциях по теме выпускной работы
\researchpublications{Данная работа не имеет публикаций.}

%% Эта команда генерирует титульный лист и аннотацию.
\maketitle{Бакалавр}

%% Оглавление
\tableofcontents

%% Макрос для введения. Совместим со старым стилевиком.
\startprefacepage

Решения, основанные на блокчейне, показывают возможность создания
распределенной сети, решающей конкретную проблему, одновременно не
требуя доверия к конкретным узлам сети и поддерживая стабильность
протокола. Блокчейн был применен в разных сферах проектирования
ПО. Особенно популярными стали решения в области криптовалют,
основанные на блокчейне, такие как Bitcoin [1], Ouroboros [3]. Также
существуют системы для распределенного хранения данных
(storj\footnote{Storj decentralized cloud storage
  \url{https://storj.io/storj.pdf}}, Permacoin[4]), голосования
(followmyvote\footnote{Follow my vote:
  \url{https://followmyvote.com/}}, bitcongress\footnote{BitCongress:
  \url{http://www.bitcongress.org/}}). Область применения широка и не
ограничивается приведенными примерами. Блокчейн -- это простое и
устойчивое решение для создания распределенной сети с большим
количеством узлов, что является ограничением BFT (byzantine fault
tolerant -- устойчивым к проблеме византийских генералов) алгоритмов
достижения консенсуса [2]. Блокчейн-решения хорошо масштабируются
горизонтально и предоставляют базу для алгоритмов, имеющих гарантии
безопасности относительно различных атак.

Программное обеспечение для организации рабочего процесса и
тайм-трекинга на первый взгляд полностью ортогонально продуктам,
основанным на блокчейне. Все рассматриваемые решения можно условно
поделить на две категории: органайзеры (google
calendar\footnote{Google calendar:
  \url{https://gsuite.google.com/learning-center/products/calendar/}},
remember the milk\footnote{Remember the milk: online to-do list and
  task management \url{https://www.rememberthemilk.com/tour/}}) и
тайм-трекеры (arbtt\footnote{Arbtt: completely automatic time tracker
  \url{https://arbtt.nomeata.de/#what}}, toggl\footnote{Toggl time
  tracker \& employee timesheet software
  \url{https://toggl.com/}}). Некоторые продукты совмещают в себе обе
функциональности: org-mode\footnote{Org-mode:
  \url{http://orgmode.org/}}, youtrack\footnote{ Youtrack:
  \url{https://www.jetbrains.com/youtrack/}}. Органайзеры оперируют в
терминах задач. Задачам можно присвоить различные временнЫе атрибуты,
вроде запланированного времени, чтобы не пропустить важную
встречу. Тайм-трекеры позволяют эффективно снимать статистику о том,
как пользователь тратит свое время. Анализ собранной статистики может
помочь найти проблемные места в организации жизнедеятельности
(например, выявлять и устранять прокрастинацию).

Однако, автору не известно ни одно решение, поддерживающее
мультипользовательский тайм-трекинг. Под этим термином подразумевается
возможность хранения не только информации о том, что группа
пользователей совместно участвовала в какой-то деятельности, но и
подтверждение этой информации от каждого участика. Эта идея
продолжается на мультипользовательские контракты. Если сервис имеет
достаточно знаний о том, какие группы пользователей должны тратить
время совместно на установленные задачи и в каком объеме, то оценка
таких соглашений может быть автоматизирована, если известны данные о
тайм-трекинге.

Блокчейн и криптовалюты предлагают прямое решение. Во-первых, идея
мультипользовательского тайм-трекинга близка к идее
мультипользовательских транзакций. Во-вторых, большинство криптовалют
реализуют в том или ином виде фреймворк для скриптинга, который
позволяет выполнять произвольные вычисления с данными на блокчейне,
что может быть эффективно использовано для распределения очков
рейтинга в рамках контракта. В-третьих, высокий рейтинг пользователей
свидетельствует о их компетентности в вопросе
договоренностей. Блокчейн представляет эффективный способ хранения
псевдо-публичной информации и дает возможности гибко настроить
механизм публикации таких данных.

Для большей наглядности того, как могут быть применена система с
поддержкой мультипользовательских контрактов и транзакций, приведем
два примера:

\begin{enumerate}
  \item Модифицированный семейный спор. Бессрочный контракт периодом в
    две недели предполагает, что на активность ``театр'' будет
    потрачено как минимум 3 часа в одной транзакции, а на активность
    ``футбол'' будет потрачено не менее четырех (в рамках
    периода). При этом контракт обязывает обоих партнеров участвовать
    в каждой из этих активностей. Прийдя на мероприятие, супруги (с
    помощью мобильного приложения) отмечают начало и тип действия, а
    по окончании каждый подтверждает эту информацию с помощью своего
    публичного ключа (опять-таки, через интерфейс мобильного
    приложения). По истечении недели умный контракт распределяет
    супругам очки рейтинга в зависимости от того, как были
    удовлетворены нужды каждого партнера. Ни один из участников не в
    силах самостоятельно подделать информацию о совместном
    времяпровождении. Поскольку партнеры пытаются максимизировать свой
    рейтинг, их стратегия должна быть построена так, чтобы избегать
    посещения мероприятий в одиночку. Любой из партнеров может
    провести псевдо-публичный аудит этой информации стороннему лицу,
    подтвердив свою способность следовать семейным
    договоренностям. Схема хорошо обобщается на полиаморические
    отношения.
  \item Фриланс биржа. Компания, которая хостит сайт для фрилансеров,
    разворачивает локальный мультипользовательский тайм-трекер для
    использования в связке с функционалом сайта. Заказчик и
    исполнитель, оформляя контракт через сайт, создают скрипт-контракт
    в блокчейне. Они договариются на 40 часов еженедельной работы в
    течение двух месяцев и на минимум два часовых митинга в течение
    каждой недели. Исполнитель отмечает свою рабочую деятельность с
    помощью интерфейса сайта и просит заказчика подтвердить транзакцию
    в конце дня, предоставляя ему сводку выполненных задач за
    день. Еженедельные митинги трекаются с обеих сторон. В случае
    невыполнения требований исполнитель теряет рейтинг, который
    привязан к его аккаунту.
\end{enumerate}

%% Начало содержательной части.
\chapter{Обзор предметной области}

\section{Децентрализованные платежные системы}

Биткоин -- первая распределенная платежная система, получившая
значительную популярность. Среди других подобных решений, биткоин
занимает 62\% всей рыночной капитализации\footnote{по данным
  \url{https://coinmarketcap.com/} на 28.04.2017}. Блокчейн-технология
для распределнного хранения состояния и вероятностный алгоритм PoW
(proof-of-work, дословно ``доказательство выполнения работы'') стали
идеальной комбинацией, которая и привела к широкому
распостранению. Существует много других решений в сфере криптовалют,
из которых значительная часть также используют блокчейн + PoW.

\subsection{Блокчейн (WIP)}

Блокчейн это очень простое решение для поддержки общего состояния
узлов в сети, обеспечивающее открытость (любой узел может
присоединиться), полную децентрализацию и отличное горизонтальное
масштабирование. Блокчейн/PoW решения обычно противопоставляются BFT
алгоритмам репликации состояния. Эта дуальность порождает
компромиссную модель, описанную в [2]: в частности, BFT алгоритмы
обеспечивают большую пропускную способность и производительность.

\subsection{Транзакции и UTXO}

Сравнение PoW и PoS. Нужно описать тут, потому что это не упоминается
в секции про Ouroboros.

\section{Ouroboros}

Ouroboros это первый протокол криптовалюты, основанный на схеме PoS и
имеющий доказанные гарантии безопасности относительно условий
стабильности цепи. Выбор этого протокола обоснован несколькими
ключевыми факторами. Во-первых, подход PoS превосходит PoW решения как
в защите от атак (TODO selfish mining, bribery attacks), так и в
экономии вычислительных ресурсов. Остальные подходы не являются
достаточно формальными и популярными одновременно и так или иначе
тратят избыток физического ресурса. Во-вторых, Ouroboros основан на
идее слоттинга -- разбиения времени на слоты одинаковой
длительности. Таким образом, блоки выпускаются в более-менее
фиксированный период времени. Это свойство не выполняется в PoW
решенях, где разница между выпущенными блоками вероятностна и основана
на текущей сложности выпуска блока -- переменная, которая в общем
случае меняется со временем. Ресурс BlockchainInfo предоставляет
возможность удостовериться, что реальное время выпуска блока в bitcoin
варьируется от 1 до более чем 30 минут.

\subsection{Общая модель протокола}

Эта подсекция описывает общую модель протокола Ouroboros.

\subsubsection{Слоттинг и параметры безопасности}

Протокол фиксирует следующие публично известные константы,
используемые для слоттинга: $startT, slotT, k, epochK$. $startT$ --
метка времени (в POSIX миллисекундах), означающая время начала работы
протокола. $slotT$ -- время одного слота в миллисекундах. $k$ это
константа безопасности, означающая количество последних неустойчивых
блоков в цепи. $epochK$ является множителем $k$, задающий эпоху. Для
удобства в дальнейшем введем еще одну переменную: $epochSlots = epochK
\times k$.

Слот -- это монотонно возрастающая функция от текущего времени
$currentT$, которая вычисляется как $getSlot(t) = (t - startT) /
slotT$, где все переменные имеют одну размерность (например,
миллисекунды). Длина слота $slotT$ должна быть больше ожидаемой
задержки на доставку среднего сообщения протокола.  Слот $j$
принадлежит эпохе $i$, если $i = \lfloor j / epochSlots \rfloor$.

Параметр безопасности $k$ определяет постоянство (persistence)
системы. Транзакция считается стабильной, если ее позиция в общей цепи
глубже, чем последние $k$ блоков.

\subsubsection{Свойства цепи транзакций}

Блокчейн Ouroboros'а спроектирован в манере близкой к большинству
имплементаций, в том числе к блокчейну биткоина. Он содержит в себе
список блоков, в котором каждый последующий (более новый) содержит в
себе хэш заголовка предыдущего, заголовок и тело. В отличии от
блокчейна биткоина, блоки Ouroboros'а делятся на два типа: главные
блоки и генезис блоки. Каждый главный блок имеет соответствующий
слот. Каждый (прошедший) слот имеет один соответствующий ему главный
блок, или не имеет вовсе.

Главные блоки несут основную функциональность транзакционного журнала
и схожи с блоками биткоина. Тело главного блока содержит список
транзакций, данные о SSC и другую относительно большую
информацию. Заголовок содержит следующие поля:

\begin{enumerate}
\item Доказательство данных тела: дататайп, содержащий информацию,
  необходимую для того, чтобы авторизовать тело блока. В частности,
  содержит дерево Меркла, построенное на списке транзакций.
\item Данные консенсуса: текущая эпоха и слот, сложность (высота
  текущего блока в общей цепи), публичный ключ лидера слота, подпись
  блока (заголовка, публичным ключом лидера слота).
\end{enumerate}

Генезис блок содержит информацию о слот лидерах. Лидер слота это
участник сети (имеющий достаточную долю средств с точки зрения
блокчейна), который единственно уполномочен выпустить блок в
конкретный слот. В течение каждой эпохи алгоритм, называющийся SSC
(shared seed calculation), собирает случайные данные для формирования
сида, который детерминированно определяет слот лидеров следующей
эпохи. Список слот лидеров и сид -- все, что находится в теле
генезисного блока. Заголовок схож с заголовком главного блока -- он
содержит в себе доказательство данных тела и эпоху. Генезис блоки
выпускаются участниками сети независимо и равны при условии того, что
участники видят один и тот же префикс блокчейна.

\subsubsection{SSC}

Ouroboros предлагает использовать вариант протокола конфиденциального
вычисления (multi-party computation, MPC) для вычисления равномерно
случайной строки, которая будет использована в процессе выбора
лидеров.

Коммитментом от случайной строки $r$ и сообщения $m$ является такое
сообщение $c = Com(r,m)$, что для спареной функции раскрытия $o =
Open(r,m)$ само $o$ является доказательством того, что $m$ --
сообщение внутри $c$. В то же время $c$ сам по себе не раскрывает
никакой информации о $m$.

Участвовать в протоколе SSC могут только те пользователи, которые
имеют достаточно средств (выше установленного порога) по состоянию на
конец предыдущей эпохи. Каждая эпоха делится на три фазы по: фаза
коммитментов ($2k$ слотов), фаза раскрытий ($4k$ слотов) и фаза
восстановления (последние $4k$ слотов). В первую фазу, которая длится
$2k$ блоков, участники SSC отсылают коммитменты. Во вторую фазу они
раскрывают свои коммитменты, отправляя раскрытия сделанных ранее
коммитментов. Третья фаза необходима для восстановления секрета при
условии, что какие-то раскрытия были потерянны. Для этого используется
дополнительный слой криптографии, обеспечивающий избыточность данных
(Verifiable secret sharing).

Все данные SSC хранятся в телах главных блоков.

\subsection{Система поощрений}

Система поощрений является набором правил, которые стимулируют
участников протокола принимать участие в выпуске блоков и SSC с
помощью присвоения им награды.

Для того, чтобы смотивировать участников принимать транзакции, каждому
слоту присваивается фиксированное количество эндорсеров. Задача
эндорсера состоит в том, чтобы собирать транзакции в сети. Лидер слота
получает подписанные данные от эндорсеров, объединяет их и кладет эти
данные в блок. Каждый участник сети может установить соответствие
между транзакциями и эндорсерами, от которых они были получены.

В систему также вводится понятие комиссии транзакции. Проверка
транзакции на валидность допускает ненулевую разницу между суммой
выходов и входов транзакции. Это значение и называется комиссией. Все
комиссии за эпоху $e$ суммируются, получая значение $S$. Протокол
предлагает равномерное распределение $S$ среди лидеров слота и
эндорсеров: участник $U$ может претендовать на долю $m * S / n$, где
$m$ это количество раз которое он был лидером или эндорсером за эпоху
$e$, а $n$ суммарное число лидеров и эндорсеров в эту эпоху. $U$
получает деньги, отправляя себе транзакцию с указанной суммой в эпоху
$e+1$.

Утверждается, что представленная стратегия распределения награды
соответствует приближенному равновесию по Нэшу, то есть любой участник
протокола, отклоняющийся от него, получает не более $1 + \epsilon$ той
награды, которую он бы точно получил, придерживаясь протокола.

\section{Улучшения и дополнительная функциональность}

Биткоин в текущей своей реализации поддерживает большое количество
дополнительных уровней протокола, не упомянутых в оригинальной
статье. Улучшения и модификации специфицируются в BIP (bitcoin
improvement proposals). Эта секция описывает некоторые из них.

\subsection{Скриптинг и P2SH}

Скриптинг или P2SH (pay-to-script-hash, дословно ``перевод на хэш
скрипта'') -- модификация криптовалюты, позволяющая пользователям
создавать адреса с гибкими, настраиваемыми условиями снятия денег. В
биткоине стандартизирована под
BIP16 \footnote{\url{https://github.com/bitcoin/bips/blob/master/bip-0016.mediawiki}}. Данная
часть опишет более общую структуру.

Протокол криптовалюты определяет поддерживаемый скриптовый язык. Он
может быть как тьюринг-полным (Plutus, [5]), так и неполным (биткоин
использует стековый язык, похожий на Форт). Пользователь создает
скрипт-валидатор, который принимает один аргумент произвольного типа и
возвращает, условно, булевое значение. Валидатор может быть
захеширован и использован как публичный адрес криптовалюты. Любой
пользователь может перевести средства на этот адрес. Для того, чтобы
перевести деньги с этого адреса, к доказательству транзакции должен
быть приложена пара из двух скриптов. Один из них -- валидатор, хэш
которого должен совпадать с адресом, с которого списываются
средства. Второй -- ридимер, генерирует входные данные для
валидатора. Чтобы проверить валидность траты средств, нужно
сгенерировать аргументы ридимером, передать их валидатору и
удостовериться, что последний вернул $true$.

Основная реализация протокола Ouroboros (Cardano
SL\footnote{https://github.com/input-output-hk/cardano-sl}) использует
скриптовый язык Plutus Ouroboros implementation (cardano-sl) uses
scripting language Plutus. Это функциональный тьюринг-полный язык со
строгой типизацией, схожий с Haskell. Скрипт-валидатор в Plutus имеет
типовую сигнатуру \lstinline|validator :: A -> Comp B|, где
\lstinline|Comp| это встроенный дататайп, изоморфный $Maybe$ с
точностью до инстанса класса монады. Он имеет два конструктора:
\lstinline|success :: B -> Comp B| и \lstinline|failure :: Comp B|.
Скрипт-ридимер в Plutus имеет тип \lstinline|Comp A|. Несложно
видеть, что поскольку $Comp$ является монадой, то проверка возможности
потратить инпут есть \lstinline|redeemer >>= validator|. Стоит
отметить, что несмотря на то, что возвращаемое значение валидатора --
это \lstinline|Comp a|, алгоритм валидации не проверяет значение
внутри конструктора, а лишь удостоверяется, что значение не является
\lstinline|failure|.

Приведем пример тривиального валидатора и ридимера.

\begin{lstlisting}[float=!h,caption={Пример пары валидатор/ридимер на Plutus}]
data Foo = { Foo | Bar }

validator : Foo -> Comp Unit {
    validator x = case x of {
        Foo  -> success MkUnit ;
        Bar  -> failure } }

redeemer : Comp Int {
    redeemer = success Foo }
\end{lstlisting}

Plutus имеет стандартную библиотеку, которая предоставляет базовые
типы данных и функции: обертки $Either$ и $Maybe$, списки и различные
придкаты. Также существует поддержка для добавления встроенных
функций. Встроенные функции могут использовать ресурсы узла сети. С
помощью них валидатор может, например, достать UTXO, относящееся к
какому-то конкретному блоку или периоду блокчейна. Фреймворк
достаточно гибок достаточно сложные операции валидации.

\subsection{Цветные монеты}

Цветные монеты (colored coins) это расширение криптовалюты,
позволяющее поддерживать разные типы сбережений. К значению ресурса
добавляется специальный тег, который наызвается цветом. Цветные
моменты были изначально стандартизированы и реализованы как настройка
над биткоином в стандарте
EPOBC\footnote{\url{https://github.com/chromaway/ngcccbase/wiki/EPOBC_simple}}
в 2012. В настоящий момент существуют еще несколько иных решений:
стандарт OAP \footnote{Open Assets Protocol
  \url{https://github.com/OpenAssets/open-assets-protocol/blob/master/specification.mediawiki}},
Coinspark\footnote{\url{http://coinspark.org/}} и
Colu\footnote{\url{https://www.colu.com/}}. Существуют виртуальные
кошельки, поддерживающие несколько стандартов
одновременно\footnote{ChromaWallet поддерживает OAP и OPOBC
  \url{http://chromawallet.com/}}. Стоит отметить, что все
перечисленные стандарты используют метаданные транзакций в биткоине
для передачи информации о цвете, а не изменяют протокол напрямую (что
привело бы к форку). Тем не менее, поддержка стандарта с изменением
протокола может быть сделана проще и эффективнее.

С интеграцией цветных монет связано несколько дополнительных
задач. Во-первых, изменяется логика валидации транзакции. Если раньше
достаточным условием было проверить, что сумма выходов не превосходит
сумму входов, то с цветными монетами существует несколько
подходов. Первый предлагает запрет на транзакции, ``красящие'' монеты
-- то есть изменяющие цвет в транзакции. В этом случае валидация
транзакции сводится к валидации сужений транзакций до каждого
цвета. Этот подход реализован во всех упомянутых
стандартах. Во-вторых, изменяется логика эмиссии ресурсов. OAP
предоставляет возможность производить эмиссию активов, удостоверяя
эмитента отдельным секретным ключом. Вместо возможности производить
чистую эмиссию можно предложить решение с покраской монет --
транзакция может менять тег средств, если она удовлетворяет правилам
покраски. Этот подход был применен при имплементации цветных монет в
RSCoin\footnote{Репозиторий RSCoin на github:
  \url{https://github.com/input-output-hk/rscoin-haskell}}: монеты
цвета $0$ могли быть покрашены в цвет $i \neq 0$ (единственное правило
перекраски). EPOBC решает проблему похожим образом, разрешая перевод
базовых монет (монет базовой криптовалюты) в любой другой цвет.

\subsection{Иерархический детерминированный кошелек}

Дополнение к протоколу биткоина под названием HD кошелек (Hierarchical
Deterministic Wallets) было предложено в рамках
BIP32\footnote{\url{https://github.com/bitcoin/bips/blob/master/bip-0032.mediawiki}}
и предлагает расширение схемы подписей
secp256k\footnote{\url{http://www.secg.org/sec2-v2.pdf}}, решающее
проблему дублирования секретных ключей на разных устройствах.

Пара ключей $Pk$ и $Sk$ (стандарт secp256k1) расширяются 32 байтами
энтропии $c$, которая называется кодом цепи (chain code), результируя
в пару расширенных ключей $Pke = (Pk, c)$ и $Ske = (Sk, c)$ по 64 байт
каждый. Затем вводятся две ключевые функции:

\begin{enumerate}
\item $CKDpriv : (Ske, i) \rightarrow Ske$. Принимает на вход расширенный
  секретный ключ $Ske$, целое число $i$ в интервале $[0..2^{32})$ и
  возвращает расширенный секретный ключ.
\item $CKDpub : (Pke, i) \rightarrow Maybe Pke$. Если $i \geq 2^{31}$, функция
  возвращает $Nothing$. Это расширение называется харденингом
  (hardening), запрещая выводить половину интервала публичных дочерних
  ключей. Если $i < 2^{31}$, возвращает расширенный публичный ключ.
\end{enumerate}

Функции $CKDpriv$ и $CKDpub$ построены таким образом, что каждая
расширенная пара $\langle Ske,Pke \rangle$, полученная с их помощью для
одного $i$, является валидной парой секретный-публичный
ключ. Возможность выводить ключи в любом другом направлении
(родительские из дочерних, секретные из публичных) отсутствует.

Стандарт реализован в Cardano SL/Ouroboros и является основным. Кроме
этого, Cardano SL поддерживает обычные ключи secp256k1 и
скрипт-адреса.

\chapterconclusion

В части были описаны решения, необходимые для построения алгоритма
тайм-трекинга поверх криптовалюты.

\chapter{Описание алгоритма}

\section{Интеграция цветных монет}

Поскольку данная работа предлагает изменение протокола Ouroboros без
поддержки обратной совместимости, поддержка цветных монет существенно
упрощается, поскольку не требует надстраивания (например, через
метаданные транзакции). Эта секция описывает различные нюансы и детали
реализации.

Первая необходимая модификация это изменение структуры данных
монеты. Добавим тег цвета типа $Int32$ к каждой монете. Монеты цвета
$basec$ считаются базовыми монетами. Тег в бинарном (сериализованное)
представлении монеты может опущен, в этом случае предполагается, что
тег равен $basec$. Также добавим два других типа монет -- $ratec$ и
$\tau c$. $Ratec$ монеты будут использоваться для представления
пользовательского рейтинга в системе, а $\tau c$ будут представлять
очки времени. Четвертая группа монет -- это монеты действия
$actc_i$. Размер этого множество можно, для простоты, взять
дополнением всего пространства цветов за вычетом упомянутых ранее, то
есть $N_a = 2^{32}-3$. ОБозначим множество $actc = \{actc_i\}_{i=1..N_a}$.

Воспользуемся схемой покраски монет, не прибегая к возможности эмиссии
активов нового цвета. Все монеты конвертируются один к
одному. Предполагается, что реальная стоимость одной монеты
пренебрежима (единица монеты в биткоине сатоши стоит 0.00001
USD). Построим ориентированный граф конвертации цветов монет $G$, где
вершины представляют собой типы монет, а ребро $uv$ принадлежит графу,
если возможна конвертация монеты типа $u$ в тип $v$. Используемый граф
конвертации будет иметь следующую форму:
\begin{align*}
  TG = \langle &\{basec, ratec, \tau c\} \cup actc,\\
  &\{basec \rightarrow ratec
  , basec \rightarrow \tau c\} \cup
  \{\tau c \rightarrow actc_i\}_{i=1..N_a} \cup
  \{actc_i \rightarrow \tau c\}_{i=1..N_a}\rangle
\end{align*}

Таким образом, он включает в себя два односторонних правила
преобразования и одно двухстороннее для каждой пары $\langle \tau c,
actc_i\rangle$.

Правило валидации транзакции должно определить, может ли выделенный
сет входов быть покрашен в цвета выхода в соответствии с $G$. Легко
заметить, что некоторые формы графов позволяют проводить линейную по
числу входов и выходов валидацию: представим граф $\{A \rightarrow B,
C \rightarrow D, E \rightarrow F, \cdots\}$ -- его валидация сводится
к валидации сужений транзакции на пары цветов.

TODO тут должен быть алгоритм валидации транзакции на абстрактном графе.

Что касается гарантий стабильности протокола Ouroboros, предлагается
поддерживать только $basec$ во всех частях алгоритма, связанных с
размером сбережений (SSC/выбор лидеров, делегация и различные пороги,
система поощрений). Легко видеть, что эта модификация не затрагивает
гарантий базового протокола:
\begin{enumerate}
\item Не существует возможности получить $basec$ монеты из любого
  другого цвета
\item Общая потеря $basec$ монет при конвертации не меняет поведения,
  потому что базовая реализация позволяет уничтожать монеты, выставляя
  ненулевую разницу между входами и выходами транзакции. Также, не
  запрещено отправлять средства на адреса с неизвестным секретным
  ключом, что эквивалентно уничтожению средств.
\end{enumerate}

\section{Система контрактов}

Контракт -- формализация способа взаимодействия пользователей как
множества обязательств, нужд и правил, а также соответствия этим
обязательствам в виде рейтинга. Главная цель этой секции состоит в
описании обязательств таким образом, чтобы, имея данные о том, какие
относящиеся к контракту действия были выполнены (и в какой мере),
пользователям можно было бы проставить соответствующий рейтинг. Этот
рейтинг должен коррелировать с усилиями пользователей в отношении
правил контракта и мотивировать участников выполнять свои
обязательства.

\subsection{Обязанности и потребности}

Представим условие контракта в виде набора обязанностей. Обязанность в
общем виде описывается следующими параметрами:

\begin{enumerate}
\item Тип деятельности: любая активность, имеющая начало и
  конец. Например, парное программирование или какой-либо вид спорта.
\item Временное ограничение: как часто и как долго деятельность должна
  иметь место. Некоторые типы деятельностей очевидным образом не
  входят в это описание из-за их случайной природы (например, условная
  деятельность ``прогулка в парке, если не идет дождь''). Тем не
  менее, ограничение по времени покрывает большую долю необходимых
  ситуаций.
\item Получатель: тот, кто определяет необходимость в совместном
  времяпровождении. Получатель -- пользователь, вместе с которым эта
  активность должна происходить. Групповые деятельности, в которых
  получателем является каждый участник, описываются множеством
  обязанностей (по одной на удовлетворения потребностей каждого). Для
  активностей без явного получателя это поле может быть
  опущено. Пример: ученик должен тратить минимум 30 часов еженедельно
  на домашние задания, тем не менее это требование родителей.
\item Поставщики: пользователи, ответственные за то, чтобы
  обязательство было выполнено. В случае невыполнения требований,
  теряют свой рейтинг. Описываются в дизъюнктивной нормальной форме
  без отрицания -- такое представление одновременно предоставляет
  необходимую гибкость и значительно упрощает валидацию при подсчете
  рейтинга.
\end{enumerate}

Формальное описание обязанности (TODO сделать красивее):

\begin{align*}
Maybe \ a &= Just \ a \ | \ Nothing \\
Andterm \ a &= a \ | \ (Andterm \ a) \wedge (Andterm \ a) \\
Term \ a &= Andterm \ a \ | \ (Term \ a) \vee (Term \ a) \\
Suppliers &= Term \ UserId\\
Receiver &= UserId\\
Timeframe &= \langle Hours, Maybe \ Times \rangle\\
Responsibility &= \langle Suppliers, Maybe \ Receiver, ActivityType, Timeframe \rangle
\end{align*}

Частым случаем является набор ответственностей, имеющих одно множество
``поставищики и получатель''. Пусть дано множество обязанностей
$R_1$. Рассмотрим $R_2 \subset R_1$, такое что для каждой обязанности
$r \in R_2$ если $r$ имеет форму $S_1\&S_2\&…\&S_n \rightarrow Receiver$. Построим ориентированный граф $G$, создав по вершине на
каждого пользователя, являющегося поставщиком или получателем в хотя
бы одной $r \in R_2$ и по ребру на $\forall r \in R_2 \langle S_i,
Receiver \rangle$ . Преобразуем $G$ в $F$: оставим вершины, но добавим
$uv$ в $F$ только если $uv \in G \wedge vu \in G$. Заметим, что если в
$R_2$ был набор пользователей $U = \{U_i\}_{i=1..N}$ и набор
обязанностей $\{U \setminus U_i \rightarrow U_i\}_{i=1..M}$ с
одним и тем же временным ограничением и типом активности, то $F$
содержит клику на множестве ребер $U$. Эти рассуждения могут быть
применены для эффективного кодирования (сериализации) набора
обязанностей.

Такое описание обязанности является достаточно широким, чтобы
покрывать большинство необходимых ситуаций и в то же время простым,
что позволяет автоматизировать процесс выставления рейтинга.

\subsection{Скрипт контракта}

Эта секция описывает схему интеграции контракта в скрипт
криптовалюты. Скрипт-адрес представляет из себя аккаунт, который
поддерживает единственную операцию снятия с себя средств --
продление. Продление контракта представляет собой транзакцию, которая
распределяет рейтинг по выделенным дополнительным адресам,
ассоциированным с этим скриптом. Чтобы инициировать контракт,
необходимо создать скрипт и перевести $basec$ монеты на его счет.

Сам код скрипта делится на две части -- встроенный код и
конфигурацию. Конфигурация -- это темплейтные поля, которые
заполняются пользователем. Она содержит в себе следующие параметры:

\begin{enumerate}
\item Период контракта $cPeriod$. Поскольку Ouroboros имеет встроенную
  поддержку слоттинга с фиксированной длиной слота, то пользователь
  может указать период в слотах, а скрипт может оперировать этим
  понятием с хорошей точностью (доставать все транзакции за выделенный
  период времени).
\item Метка старта контракта $cStart$. Для того, чтобы контракт мог
  выделить последний период, ему необходимо знать слот, в котором
  появилась последняя транзакция продления (это можно сделать с
  помощью встроенных в скриптовый язык функций). Но до первого
  продления эта информация не существует. Метка старта представляет из
  себя индекс слота и используется ровно в этом случае.
\item Список публичных ключей участников контракта $cUsers$.
\item Набор обязательств $cResponsibilites$, индексированный ключами
  из $cUsers$.
\item Расширенный публичный HD ключ $cPke$, который будет в дальнейшем
  использован для построения адресов активностей.
\end{enumerate}

Отдельно необходимо выделить, что имея темплейт скрипта, мы всегда
можем проверить, подходит ли под него конкретная
инстанциация\footnote{Эта возможность используется в биткоин --
скрипты, не подходящие под фиксированный список темплейтов, не
распостраняются клиентами TODO LINK}. Это дает возможность выявлять и
игнорировать модифицированные скрипты.

Таким образом, внутреннее состояние контракта характеризуется, в
первую очередь, его последним периодом $cDelta = [siLast \cdots
siLast+cPeriod]$, где $siLast$ -- функция, возвращающая слот
последней транзакции продления или $cStart$, если транзакции
отсутствуют. Для ее реализации необходимо добавить в plutus поддержку
встроенного примитива $searchTx$, который с помощью ресурсов узла сети
произведет поиск по транзакцям, соответствующим предикату. В случае
$siLast$ необходимо найти последнюю транзакцию перевода средств с
публичного адреса скрипта. После того, как $cDelta$ определен, скрипт
производит поиск всех трекинг-транзакций в этом периоде, считает
рейтинг для каждого пользователя и удостоверяется, что распределение
рейтинга в транзакции соответствует подсчитанному.

\subsection{Схема транзакций}

Транзакция активности несет в себе информацию об одном виде активности
и том, сколько каждый ее участник потратил на нее времени. Обозначим
за $Users = \{U_i\}_{i=1..N}$ множество пользователей, участвующих в
деятельности типа $ActivityType = actk_k$ для некоторого $k$. $Spent_i
= \langle \tau S_i, \Delta_i \rangle$ -- временной промежуток (пара из
начала и длительности, в POSIX миллисекундах и минутах
соответственно), соответствующий реальной активности пользователя
$U_i$. Тогда определим пред-транзакцию как $ActionTxPre = \langle
Users, \{Spent_i\}_{i=1..N}, ActivityType \rangle$. Эта модель
удовлетворяет требованиям, предъявляемым контрактом, но несет в себе
избыточную информацию о реальном времени. Необходимо знать две вещи:
когда произошла транзакция (эта информация выводится из слота блока, в
который транзакция попадет) и сколько времени пользователи потратили
на каждого другого участника транзакции в ее же рамках. Заметим, что
последнее замечание критично ввиду потенциально нетривиального
пересечения отрезков из множества $\{Spent_i\}$.

Чтобы построить реальную транзакцию активности $ActionTx$, нам
потребуется превратить $ActionTxPre$ в два списка -- входы и выходы
транзакции. Пусть $UPk_i$ -- публичный ключ $U_i$. Введем функцию
$CActOut(C, {U_{i_j}}) : Pk$, которая по контракту и непустому
подмножеству пользователей $\{U_{i_j}\} \in Users$ возвращает
публичный ключ. Тут и далее публичный ключ и адрес будут
использоваться взаимозаменяемо, если это не вызывает
неоднозначности. $CActOut$ инъективна и имеет конечное множество
значений размера $2^N = \sum_{i=1}^N{\binom{N}{i}}$. Эта функция
используется для того, чтобы ассоциировать контракт с адресами
активности, на которых будут накапливаться $actc_i$.

Пусть $ESet = \{Ev_i\}_{i=1..k}$ есть множество всех событий из
интервалов в множестве $\{Spent_i\}$, где событием является начало или
конец промежутка. Легко видеть, что для каждого выделенного интервала
из $\{\langle Ev_i, Ev_{i+1} \rangle\}_{i=1..k-1}$ есть ровно одно
активное подмножество $U \subset Users$. Для каждого $U_i$ и его
$Spent_i$ поделим $Delta_i$ на множество интервалов $\{Delta_{i,j}\}$,
разбивая в соответствии с $ESet$. Построим множество соответствия
$Compl_i$: по каждому интервалу $Delta_{i,j}$ получим $CActP_k$ (с
помощью функции $CActOut$) и объединим все $\{\langle Delta_{i,j},
CActP_{k,j}\rangle\}$, имеющие один публичный ключ, суммируя по
интервалам, чтобы избежать дубликатов. Обозначим это множество пар за
$Compl_i$. Таким образом, $Compl_i$ содержит информацию о том, сколько
времени $U_i$ взаимодействовал с каждым другим участником транзакции.

Эта процедура проводится для каждого пользователя. Далее
$\{Compl_i\}_{i=1..N}$ может быть напрямую преобразовано в
транзакцию. $U_i$ определяет $\sum(Delta_{i,s})$ $tauc$ как вход своей
подтранзакции, и для каждого $\langle Delta_{i,j}, CActP_{k,j}\rangle
\in Compl_i$ выставляет выход в соответствующее $Delta_{i,j}$
количество монет, адрес $CactP_{k,j}$ и цвет монеты
$ActivityType$. Все подтранзакции объединяются в одну, конкатенируя
входы и выходы соответственно. В конечном счете, $TxAction$ имеет
$\langle UPk_i, UDelta_i, \tau c\rangle$ на входах для каждого
участника, а на выходах $\langle Delta_{i,j}, CActP_{k,j},
ActivityType\rangle$. Мнемонически, участник разбивает свои временнЫе
ресурсы $\tau c$ на ресурсы действия.

Валидность транзакции относительно балансов очевидна по построению,
так как сумма входных монет для каждой подтранзакции определяется
через сумму выходов и выполняется правило перекрашивания $\tau c
\rightarrow ActionType$. После преобразования, $TxAction$ не содержит
излишней информации о метках времени.

\section{Функционал адресов активностей}

Инстанциируем функцию $CActOut(C, U)$, использованную ранее для вывода
множества публичных ключей из контракта и подмножества его
пользователей. Одним из требований к ней является возможность
генерировать парные (к публичным) секретные ключи. Это поможет
переиспользовать $actc_i$, конвертируя их обратно в $\tau c$. Также,
хотелось бы минимизировать информацию, идентифицирующую эту функцию в
скрипте, поскольку валидаторы хранятся в блокчейне явно. Наивный
подход хранения списка публичных адресов является избыточным по
памяти. Решение, предлагаемое в этой главе, заключается в
использовании расширенных HD ключей.

Выделим среди пользователей контракта одного (мастера) и предположим,
что он владеет расширенным секретным ключом $CSke$. Публичная
компонента этого ключа $CPke$ находится в конфигурации
валидатора. Определим функцию $CactOut(C,U)$ следующим образом:

\begin{enumerate}
\item Занумеруем все сочетания на $Users$ с помощью $userEnum :
  2^{Users} \rightarrow \mathbb{N} / 2^N$. $userEnum$ может быть
  определена тривиально. Представим $U_i$ с помощью целого $i \in
  \mathbb{N}$, а каждое $Users_j \subset 2^{Users}$ в виде
  отсортированного списка его элементов, представленных натуральными
  числами. Тогда порядок на $Users_j$ может быть определен
  лексикографически.
\item $CActOut(C,Users_j) = CPke/userEnum(Users_j)$. Напомним, что
  запись $Pke/i$ означает выведенный из $Pke$ дочерний публичный ключ
  номер $i$ .
\end{enumerate}

\section{Функция распределения рейтинга}

Функция распределения рейтинга
\[RateAssign(\{Responsibility_i\},{ActionTx_i}, U_i) : Rating\]
принимает на вход контракт, данные транзакций и пользователя,
возвращающет рейтинг этого пользователя. Рейтинг является вещественным
числом, определяющим соответствие действий пользователя его
обязанностям. Область значений $RateAssign$ предлагается принять
равной $[0,cPeriod]$ по нескольким причинам. Во-первых, $cPeriod$ в
любой установке дает хороший баланс между реальной стоимостью монет
$ratec$ (низкая) и точностью приближения вещественного числа
натуральным (предполагая, что длительность слота не превышающет пяти
минут). Во-вторых, $cPeriod$ пропорционален временнОй длительности
контракта и наглядно характеризует кумулятивный рейтинг пользователя.

Определим свойства функции распределения рейтинга $RateAssign$
(требуют большего формализма):

\begin{enumerate}

\item Заполнение: если для каждой обязанности $R$, в которой $U_i$
  выступает поставщиком, в период контракта верно, что $R$ была
  полностью выполнена, $U_i$ получает максимальный рейтинг.

\item Аксиома болвана: пользователю, не принимающему участие ни в
  одной активности (при этом имеющему ненулевые обязательства),
  присваивается нулевой рейтинг.

\item Участие пользователя в обязанности, которая не была выполнена
  полностью, влечет снижение его рейтинга. Вхождение пользователя в
  несколько $\vee$-кейсов не снижает его рейтинг больше, чем если бы
  он входил только в один.
\end{enumerate}

Представим общую структуру алгоритма $RateAssign$. Для каждой
обзанности с ненулевым получателем $U_{rec}$ определим:

\begin{align*}
hoursLoss &= 1 - \min(\frac{spentMinutes}{totalMinutes}, 1)
\\
timesLoss &=
\begin{cases}
          hoursLoss, & \text{if } totalTimes \text{ is Nothing} \\
          1 - \min(\frac{spentTimes}{totalTimes}, 1), & \text{otherwise}
\end{cases}
\\
totalLoss &= \frac{hoursLoss + timesLoss}{2}
\end{align*}

Значение $totalLoss$ представляет собой взвешенную долю
неудовлетворенности обязательства. Далее необходимо разделить ее между
пользователями. Для начала выделим всех участников контракта, которые
входят в каждый дизъюнкт $Users_0$. Объединим остальных в множество
$Users_1$. Поделим штраф поровну между этими категориями, присвоив
потерю рейтинга каждому из $Users_0$ и поделим штраф на $Users_1$
поровну между ее участниками.  Усредним потерю рейтинга по всем
обязанностям и домножим на $cPeriod$.

Отметим, что предложенная реализация $RateAssign$ достаточно наивна и
не учитывает реальное отношение усилий, приложенными участниками для
удовлетворения обязанности.

\section{Транзакция продления и поток активов}

Единственный способ, с помощью которого с контракта можно переводить
средства, есть транзакция продления контракта. Она состоит из двух частей:

\begin{enumerate}
\item Весь объем монет типа $actc$, которые участвую в трекинговых
  транзакция, накапливаются на адресах $CActOut(C,Users_i)$. Эти
  средства могут быть переиспользованы, поэтому они являются входами
  транзакции. Соответствующий выход может иметь любой адрес и тип
  выходных средств $\tau c$. Предполагается, что мастер контракта
  переводит эти средства себе, либо другим участникам для дальнейшего
  оборота.
\item Плата за продление в $basec$. Эти монеты переходят на
  спциализиованные рейтинговые адреса ${CRewPk_i}_{i=1..N}$. Также, эта часть
  транзакции должна быть использована для оплаты самой транзакции.
\end{enumerate}

Рейтинговые адреса формируются из публичных ключей, которые выводятся
из контракта и публичного ключа пользователя. Они не имеют
соответствующих секретных ключей намеренно, что гарантирует
невозможность снять $ratec$ со счета и скрыть статистику. В качестве
реализации можно взять $CRewPk_i = Blake2b512(CPk || UPk_i)$, где
$Blake2b512$ это хэш-функция из семейства
blake2\footnote{\url{https://blake2.net/blake2.pdf}} с размером
хэш-пространства $2^{512}$ (чтобы совпадать со схемой расширенных HD
ключей).

Опишем полную схему потока активов.

\begin{enumerate}
\item Группа пользователей создает скрипт, добавляя в темплейтные поля
  конфигурационные параметры. Один пользователь назначается мастером:
  он генерирует $CSke$ и добавляет $CPke$ в конфигурацию. Любой
  участник переводит $basec$ монеты на счет $CPk$, чтобы инициировать
  контракт.
\item Первым периодом контракта является $[cStart \cdots
  cStart+cPeriod]$. Пользователи самостоятельно меняют $basec$ на
  $\tau c$ (или получают $\tau c$ любым другим способом) и отмечают
  свои действия в рамках периода контракта с помощью транзакций
  активности.
\item Начиная со слота $cStart + cPeriod + 1$ мастер может продлить
  контракт. Он создает транзакцию продления, собирая информацию о
  действиях за последний период, выставляет соответствующую плату за
  транзакцию, добавляет подписи для $CActOut$ ключей (с помощью
  $CSke$) и отправляет транзакцию в сеть.
\item Узел сети, проверяющий транзакцию, произведет проверку балансов
  и расчет платы транзакции. За этим следуют проверки валидности всех
  потраченных входов транзакции. $CActOut$ входы будут проверены с
  помощью приложенных мастером подписей, затем узел запустит валидатор
  (с пустым ридимером).
\item Валидатор имеет доступ к блочейну и текущей обрабатываемой
  транзакции. Он производит следующие проверки:
  \begin{enumerate}
  \item Выводит последний период (с помощью $siLast$ и $cPeriod$) и
    проверяет, что текущий период больше правой границы интервала.
  \item Транзакция продления подходит под общий темплейт: два типа
    входов ($CPk$ и $CActOut_i$), два типа выходов ($CRewPk_i$ адреса,
    типа выхода $ratec$; $\tau c$ тип выхода на всем остальном).
  \item Входы соответствующие адресам активностей, в точности
    совпадают со всем множеством входов, использованных в последний
    период. Валидатор имеет возможность производить поиск по
    транзакциям (с помощью реализованного узлом примитива), поэтому
    проверка сводится к сопоставлению множеств.
  \item Распределение наград на выходе транзакций соответствует
    подсчитанному валидатором (с помощью $RateAssign$).
  \end{enumerate}
\item Транзакция продления попадает в блокчейн, распределяя очки
  рейтинга. Транзакция продления начинает новый период. Пользователи
  могут продолжить сценарий со второгу пункта.
\end{enumerate}

\section{Формирование пре-транзакций}

Для того, чтобы транзакции активности могли попасть в блокчейн, они
должны быть подписаны каждым участником. Стандартные подходы для сбора
подписей в мультипользовательских транзакциях в основном делятся на
ручные (передача подписи текстом или с помощью QR кода, например в
клиенте
electrum \footnote{\url{http://docs.electrum.org/en/latest/multisig.html}})
и на автоматические, с использованием централизованного пула
подписей\footnote{Централизованный сбор подписей в кошельке coinkite:
  \url{http://blog.coinkite.com/post/102291566521/bitcoin-multisig}}. Подход,
необходимый для формирования $PreActionTx$ и последующего сбора
подписей для $ActionTx$, более комплексный. Разделим функционал на две
части: логика и транспорт. Под логическим слоем подразумевается API
приложения. Транспорт -- физическая реализация. Мы рассмотрим
возможные решения транспорта, связанные с централизованными серверами
и децентрализованной сетью.

\subsection{Логический слой}

Главная идея этой части функционала -- предоставить пользователям
интерфейс для создания пре-транзакции и автоматизировать сбор подписи
после. Поэтому, имеет смысл разделить всю логику на две части соответственно.

Логика составления пре-транзакции со стороны транспорта предоставляет
следующие методы:
\begin{enumerate}
\item $StartActivity(IPk, {UPk_i}, actc_k, CInfo, desc) : txid$. Этот
  вызов инициирует пустую пре-транзакцию. Иницииатор прикладывает свой
  публичный ключ, публичные ключи других участников и подписывает эту
  информацию своим ключом. Также он прикладывает информацию о типе
  активности и информацию о контракте (которая в общем случае
  неизвестна узла). Вызов возвращает идентификатор пре-транзакции.
\item $ChangePermissions(args) : ()$. Вызов изменяет настройки
  доступа к транзакции. Мастер может приглашать или удалять
  приглашенных участников.
\item $SearchTx(args) : results$ производит поиск по пре-транзакциям,
  удовлетворяющим требованиям заданным в $args$. Предполагается
  поддержка поиска по публичному ключу приглашенного участника, другим
  участникам или по описанию.  Возвращает всю информацию о
  пре-транзакции, известную на текущий момент. Формат ответа
  варьируется в зависимости от запроса -- может быть возвращен список
  пре-транзакций, одна пре-транзакция либо сконвертированная
  транзакция со списком подписей.
\item $PreTxStart(txid, UPk_i, time) : Bool$ отмечает начало
  временного интервала для пользователя. Возвращает значение,
  свидетельствующее об успехе операции.
\item $PreTxExit(txid, UPk_i, time) : Bool$ завершает временной
  интервал для указанного пользователя. Возвращает маркер успеха.
\end{enumerate}

Полностью законченные пре-транзакции автоматически конвертируются в
транзакции без подписей и могут быть индексированы с помощью
$SearchTx$. API для сбора подписей состоит из единственного метода
$UploadSignature(UPk_i, \sigma, txid)$, который загружает подпись
участнка. Существующие подписи могут быть получены также с помощью
метода $SearchPreTx$. Любой участник отправляет готовую транзакцию с
подписями в сеть.

Этот слой логики также удаляет всю информацию, находящуюся в
транспортном слое слишком долго, снижая нагрузку на транспорт и
ограничивая время одной пре-транзакции. Для того, чтобы избежать
перегрузки сервиса, транспорт может быть либо приватным, либо быть
синхронизированным с блокчейном и допускать к использованию только
участников, имеющих достаточно средств (подход, применяемый в
нескольких компонентах Cardano SL).

\subsection{Транспортный слой}

Реализация централизованного сервера прямолинейна, но имеет очевидный
недостаток в виде необходимости доверять серверу и низкой
отказоустойчивости. С другой стороны, наивное распределенное решение с
помощью DHT (которое можно было бы реализовать поверх существующей p2p
сети) подвержено sybil атаке. Сценарий атаки заключается в том, что к
сети подключается большое количество вредоносных узлов. В случае с DHT
это может существенно снизить процент реплицированных данных у
остальной части сети, что приведет к их потере, если весь вредоносный
кластер уничтожит свою часть DHT локально.

Тем не менее, эта проблема может быть решена с помощью предпосылок,
которыми пользуется Ouroboros -- ни один участник сети не имеет
достаточно большого количества средств на счету. Исходя из этого,
предлагается построить модификацию Kademlia DHT, в которой могут
участвовать только узлы сети, имеющие количество средств выше
заданного порога, и степень дубликации репликации данных соотносится с
константами протокола. TODO

\chapterconclusion

В части был представлен обзор предлагаемого протокола.

%% Макрос для заключения. Совместим со старым стилевиком.
\startconclusionpage

Алгоритм хороший и примеры интересные.

\printmainbibliography




\end{document}
