%\documentclass{beamer}
\documentclass[11pt,handout,pdf,hyperref={unicode}]{beamer}
\setbeamertemplate{navigation symbols}{}

\usepackage[T2A]{fontenc}
\usepackage[utf8]{inputenc}
\usepackage[english,russian]{babel}
\usepackage[style=authoryear]{biblatex}
\usetheme{Warsaw}

\addtobeamertemplate{navigation symbols}{}{%
    \usebeamerfont{footline}%
    \usebeamercolor[black]{footline}%
    \hspace{10em}%
    \insertframenumber/\inserttotalframenumber
}

%Information to be included in the title page:
\title[Тайм-трекинг на блокчейне]{Тайм-трекинг пользовательских действий на распределенной блокчейн-цепи}

\begin{document}

\author[Михаил Волхов, M3438]{
  Студент: Михаил Волхов, M3438\\
  Руководитель: Штукенберг Д.Г., тьютор кафедры КТ \\
  Рецензцент: Чепурной А.И, специалист, IOHK Research
}
\institute{Кафедра Компьютерных Технологий \\ факультет Информационных Технологий и Программирования \\ Университет ИТМО, Санкт-Петербург}
\date{2017}

\frame{\titlepage}

\section{Введение и цели}

\subsection{Блокчейн и криптовалюты}

\begin{frame}
  \frametitle{Блокчейн}

  Блок -- это хедер с хэшом и какие-то данные. Блоки в цепочку
  выстраиваются. Разные алгоритмы консенсуса (pow/pos).
  \begin{itemize}
  \item Открытость
  \item Свобода
  \item Веганство
  \item Лучше чем BFT
  \end{itemize}
\end{frame}

\begin{frame}
  \frametitle{Криптовалюта}

  Можно отправлять денежки. Транзакция -- это входы и выходы. Адреса
  -- это публичные ключи (работают, потому что ассиметричная схема
  подписей).
\end{frame}

\begin{frame}
  \frametitle{Скриптинг и смарт-контракты}

  Такие адреса, что можно с них денежки хитро снимать. Привести пример аукциона из Hawk и что-нибудь из ethereum.
\end{frame}

\begin{frame}
  \frametitle{Цветные монетки}

  Дополнительная информация. Круто!
\end{frame}

\subsection{Тайм-трекинг}

\begin{frame}
  \frametitle{Органайзер против тайм-трекера}

  Почему гугл документ -- это не совсем то же, что и arbtt
\end{frame}

\begin{frame}
  \frametitle{org-mode и статистика}

  Показать, как там трекать и свою статистику
\end{frame}

\begin{frame}
  \frametitle{Мультипользовательский тайм-трекинг}

  Почему то, что в YT -- это не очень.
\end{frame}

\subsection{Цели и задачи}

\begin{frame}
  \frametitle{Идея}

  Написать тайм-трекер, который умеет в мультипользовательские
  транзакции, обязанности и рейтинг.
\end{frame}

\section{Проделанная работа}

\subsection{База и модификации}

\begin{frame}
  \frametitle{Ouroboros и слоттинг}
  Почему именно ouroboros.
\end{frame}

\begin{frame}
  \frametitle{Граф покраски и валидация }

  Показать, чо за граф, как транзакцию валидировать через поток.
\end{frame}

\begin{frame}
  \frametitle{Реальная инстанциация}

  Предложил реальную схему, показал как и зачем. Показал, что ничего
  не сломается.
\end{frame}

\subsection{Система контрактов}

\begin{frame}
  \frametitle{Контракт, обязательства}

  Формализовал это.
\end{frame}

\begin{frame}
  \frametitle{Выставление рейтинга}

  Написал это через вектор Шепли, получилось прикольно и какие-то
  свойства есть.
\end{frame}

\begin{frame}
  \frametitle{Общая схема}

  Про скрипт и транзакцию обновления.
\end{frame}

\subsection{Инфраструктура}

\begin{frame}
  \frametitle{Проблема и логический слой}

  Что вообще за проблема, и вкратце про API.
\end{frame}

\begin{frame}
  \frametitle{Транспортный слой}

  Обзорчик на то, что можно и почему, показать какие проблемы возникают.
\end{frame}

\section{Результаты}

\begin{frame}
  \frametitle{Результаты}

  Спроектировал решение, получилось хорошо, всем требованиям удовлетворяет (показать).
\end{frame}

\section{}

\begin{frame}

  \begin{center}
    \Huge Вопросы?
  \end{center}

\end{frame}

\end{document}
